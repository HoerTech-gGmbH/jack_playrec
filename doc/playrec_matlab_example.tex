%%% This file is part of the Open HörTech Master Hearing Aid (openMHA)
%%% Copyright © 2019 HörTech gGmbH
%%%
%%% openMHA is free software: you can redistribute it and/or modify
%%% it under the terms of the GNU Affero General Public License as published by
%%% the Free Software Foundation, version 3 of the License.
%%%
%%% openMHA is distributed in the hope that it will be useful,
%%% but WITHOUT ANY WARRANTY; without even the implied warranty of
%%% MERCHANTABILITY or FITNESS FOR A PARTICULAR PURPOSE.  See the
%%% GNU Affero General Public License, version 3 for more details.
%%%
%%% You should have received a copy of the GNU Affero General Public License, 
%%% version 3 along with openMHA.  If not, see <http://www.gnu.org/licenses/>.

% Latex header for doxygen 1.8.11
% adapted for openMHA
\documentclass[11pt,a4paper,twoside]{article}

% Packages required by doxygen
\usepackage{fixltx2e}
\usepackage{calc}
\usepackage{openMHAdoxygen}
\setlength{\headheight}{13.6pt}
\usepackage[export]{adjustbox} % also loads graphicx
\usepackage{graphicx}
\usepackage[utf8]{inputenc}
\usepackage{makeidx}
\usepackage{multicol}
\usepackage{multirow}
\PassOptionsToPackage{warn}{textcomp}
\usepackage{textcomp}
\usepackage[nointegrals]{wasysym}
\usepackage[table]{xcolor}

% Font selection
\usepackage[T1]{fontenc}
\usepackage{helvet}
\usepackage{courier}
\usepackage{amssymb}
\usepackage{sectsty}
\usepackage{textcomp}
\renewcommand{\familydefault}{\sfdefault}
\allsectionsfont{%
  \fontseries{bc}\selectfont%
  \color{darkgray}%
}
\renewcommand{\DoxyLabelFont}{%
  \fontseries{bc}\selectfont%
  \color{darkgray}%
}
\newcommand{\+}{\discretionary{\mbox{\scriptsize$\hookleftarrow$}}{}{}}

% Headers & footers
\usepackage{fancyhdr}
\pagestyle{fancyplain}
\renewcommand{\sectionmark}[1]{%
  \markright{\thesection\ #1}%
}
\fancyhead[LE]{\fancyplain{}{\bfseries\thepage}}
\fancyhead[CE]{\fancyplain{}{}}
\fancyhead[RE]{\fancyplain{}{\bfseries\leftmark}}
\fancyhead[LO]{\fancyplain{}{\bfseries\rightmark}}
\fancyhead[CO]{\fancyplain{}{}}
\fancyhead[RO]{\fancyplain{}{\bfseries\thepage}}
\fancyfoot[LE]{\fancyplain{}{}}
\fancyfoot[CE]{\fancyplain{}{}}
\fancyfoot[RE]{\fancyplain{}{\bfseries\scriptsize \copyright{} 2019 H\"orTech gGmbH, Oldenburg }}
\fancyfoot[LO]{\fancyplain{}{\bfseries\scriptsize \copyright{} 2019 H\"orTech gGmbH, Oldenburg }}
\fancyfoot[CO]{\fancyplain{}{}}
\fancyfoot[RO]{\fancyplain{}{}}












% Indices & bibliography
\usepackage{natbib}
\usepackage{tocloft}
\setcounter{tocdepth}{2}
\setcounter{secnumdepth}{4}
\addtolength{\cftsubsecnumwidth}{5pt}
\usepackage{fancyvrb}


\RecustomVerbatimCommand{\VerbatimInput}{VerbatimInput}%
{fontsize=\footnotesize,
 %
 frame=lines,  % top and bottom rule only
 framesep=2em, % separation between frame and text
 rulecolor=\color{Gray},
 %
 label=\fbox{\color{Black}data.txt},
 labelposition=topline,
 %
 commandchars=\|\(\), % escape character and argument delimiters for
                      % commands within the verbatim
 commentchar=*        % comment character
}












\makeindex

% Custom commands
\newcommand{\clearemptydoublepage}{%
  \newpage{\pagestyle{empty}\cleardoublepage}%
}

\usepackage{caption}
\captionsetup{labelsep=space,justification=centering,font={bf},singlelinecheck=off,skip=4pt,position=top}

\setlength\parindent{0pt}
\usepackage{hyperref}
\usepackage[hang,flushmargin]{footmisc}
\usepackage[margin=1in]{geometry}
\usepackage{color}
\usepackage{subcaption}
\usepackage{fancyvrb} %Für Rahmen um Code Boxen
\usepackage{listings}


\lstdefinestyle{customc}{
language = tcl,
commentstyle=\color{orange},
  %columns = flexible
  basicstyle = \ttfamily,
  showstringspaces = false,
  numbers=left,
  numberstyle=\tiny,
  frame = single
}

\lstset{escapechar=@,style=customc}








\begin{document}
\pagenumbering{arabic}

\textcolor{orange}{\textbf{Installation Instructions:}} \\ \\
\small{\url{https://github.com/HoerTech-gGmbH/jack\_playrec/blob/master/INSTALLATION.md}}



\section{Minimal Example - Usage of Matlab Function "jack\_playrec" to Communicate with JACK Server}

\textcolor{orange}{\textbf{The purpose of this minimal example is to demonstrate how jack\_playrec can be used to send audio data (created using simple Matlab functions) to a JACK client using a JACK server. In this example a simple sinusoidal signal is generated and send to "system:playback\_1".}} 

\subsection{Preparations}
\begin{enumerate}
\item Start a JACK server (sample rate = 44100, Frames/Period = 256)
    \item \textbf{Start Matlab} 
    \item Add the path of the corresponding Matlab functions by: \\

\textcolor{orange}{\textbf{Windows}} 
\begin{verbatim}
addpath('C:\Program Files\jack_playrec\mfiles')
\end{verbatim}   
\textcolor{orange}{\textbf{Linux}} 
\begin{verbatim}
addpath('/usr/lib/jack_playrec/')
\end{verbatim} 
\textcolor{orange}{\textbf{MacOS}} 
\begin{verbatim}
addpath('/usr/local/lib/jack_playrec/')
\end{verbatim} 
    
\subsection{Example Code}
You can now copy the following lines of code into the Matlab/Octave command windows. This simple code below shows a minimal example of how to send an audio signal to a JACK client (here system:playback\_1). 

\begin{verbatim}
sampling_rate = 44100; %needs to fit sample rate of JACK server
frequency = 440; %frequency in Hertz
signal_duration = 5; %signal duration in seconds
%define time array 
t= 0:(1/sampling_rate):(signal_duration-1/sampling_rate);
%define sine function
sine = transpose(sin(2*pi*frequency*t));
%play generated signal using
%jack_playrec(input signal,'output',name of output channel)
jack_playrec(sine,'output',{'system:playback_1'});
\end{verbatim}  

\end{enumerate}





\end{document}

%%% Local Variables: 
%%% mode: latex
%%% TeX-master: "openMHA_application_manual"
%%% indent-tabs-mode: nil
%%% coding: utf-8-unix
%%% End:
